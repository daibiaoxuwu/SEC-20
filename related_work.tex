\section{Related Work}
\label{sec-related-work}
\subsection{Shoulder Surfing}
With the arrival of the information era, privacy issues are becoming increasingly prominent. Smartphone screen privacy, the concern of our smartphones being observed by strangers in public areas, or shoulder surfing, have been studied heavily in recent years. Surveys of shoulder surfing provides evidence of this behavior in real world \cite{eiband2017understanding} \cite{goucher2011look}, often more efficient than expected\cite{kwon2013covert}. Various techniques and systems have been designed to mitigate this threat. Some systems sense malicious passers-by and hide information \cite{brudy2014anyone} or warn the user\cite{saad2018communicating}; others modify the user interfaces to create honeypots (for passwords)\cite{chakraborty2014tag}, confuse unauthorized parties\cite{wiedenbeck2006design}, or making the interactions invisible\cite{kumar2007reducing} or unreadable from a distance\cite{Chun2019Keep}. Most of the works designing defenses against shoulder surfing assume that the attacker is a casual passer-by, taking occasional peeks with his/hers naked eye, as is the case most of the time\cite{eiband2017understanding}. However, it is the malicious ones, although few, that can utilize the information (passwords, business correspondence, etc.) they obtained to do real harm. On the other hand, the threat model of tool-assisted shoulder surfing is also studied. Schaub, et, al. observed the susceptibility of shoulder surfing from the naked eye\cite{schaub2012password}. Maggi, et al. designed an automatic shoulder surfing threat model, observing the target smartphone with a camera\cite{maggi2011poster}, but without the help of SR techniques, this method can only function when the attacker is standing at close range, which is a barely practical scenario. With the development of smartphone camera, processing ability, and SR technology over the years, we propose a stronger shoulder surfing system in which attackers can successfully obtain information while keeping a distance to avoid suspicion, and prove that this threat is eminent in our daily life.

\subsection{Super Resolution}
Image super-resolution is the process of reconstructing an image with a higher spatial resolution. Single image super-resolution techniques accept a single low-res image as input, and based on its structural pattern, self-similarity \cite{suetake2008image}, or previous knowledge of the genre of the image, deduces missing information and reconstructs the missing pixels to form a sharp, high-res image. On the other hand, multiple image super-resolution techniques work on a set of pictures on the same scene, e.g. multiple snapshots captured by the camera of a smartphone, successive images from a satellite, or adjacent frames on a video clip. These algorithms collect extra data from slight differences of these redundant images, often exhibiting better performance than single image super-resolution.

Over the last two decades a variety of techniques have been proposed for the multiple image super-resolution problem. Early works focus on the analysis of the images on spatial or frequency domains. One of the widely known spatial domain methods is the Shift-Add algorithm \cite{farsiu2003robust}, which functions by maximizing the pixel-wise possibility of high-res image generating low-res images. On the other hand, frequency domain approaches focus on the Fourier transform of the images. By assuming that the high-res scene is band-limited, frequency domain algorithms can reconstruct high-frequency components from patterns of the low-frequency components, removing bands of noise and blur effects at the same time. A great variety of techniques have been proposed, however, due to the nature of the problem, these algorithms are all tailored for their application: natural photos, scanned text, satellite imaging, biometrics, etc. and achieve best performance only in their own field.

More recent works of Super Resolution are often based on deep learning approaches. In 2016, SRCNN \cite{dong2015image} was developed from CNN replacing pooling layers with upsampling layers, achieving notably better performance than previous approaches. This work focuses on single image super-resolution, as the neural network accepts a single image as input. Generative Adversarial Networks (GAN) were also used in this era\cite{ledig2017photo}, generating photo-realistic images, with less artifacts and more genuine-looking details pleasing to the human eye, though not always accurate against the real scene. When published, the above works were limited to single input images, and most multi-image SR tasks are variations of them. The most common variation is video SR\cite{shi2016real, kappeler2016video}, with a structure similar to SRCNN, accepting complete video clips as input, and functioning on the similarity between frames. The sequentiality and consistency between adjacent frames makes it easy for the convolution networks to comprehend, and by modifying the 2d convolution layers to 3d convolution\cite{caballero2017real}, modifying the dataflow to merge neighboring frames among the network layers\cite{huang2017video}, or recurrently processing the frames under the guidance of the output of the previous frame\cite{sajjadi2018frame}, the task can be completed easily. Other works face image groups without consistency or sequential information, e.g. satellite images. To achieve appreciable results, most works choose hybrid methods, solving the multi-image SR problem with multiple single-image SR procedures. They commonly function by merging the results of single image SR algorithms to efficiently utilize input information\cite{kawulok2019deep}, or building a multi-image network to create a course view and support it with single image SR networks\cite{8834937}. These methods expect at least some degree of information to be extracted from each of the single frames with single image SR methods. However, because of the extreme blurriness of the photos and the lack of consistency between frames, the methods mentioned above are not suitable for our application and we need a novel approach.