\section{Related Work}
\label{sec-related-work}

\subsection{Shoulder Surfing}
With the arrival of the information era, privacy issues are becoming increasingly prominent. Smartphone screen privacy, the concern of our smartphones being observed by strangers in public areas, or shoulder surfing, has been studied heavily recently~\cite{eiband2017understanding,goucher2011look,kwon2013covert}. To mitigate this threat, some systems hide the information~\cite{brudy2014anyone} or warn the user~\cite{saad2018communicating} once sensing malicious passers-by; others modify the user interfaces, including creating honeypots (for passwords)~\cite{chakraborty2014tag}, confusing unauthorized parties~\cite{wiedenbeck2006design}, and making the interactions invisible~\cite{kumar2007reducing} or unreadable from a distance~\cite{Chun2019Keep}. Most of the works assume that the attacker is a casual passer-by, taking occasional peeks with the naked eye, as is the case most of the time~\cite{eiband2017understanding,wiedenbeck2006design}. Given the assisted equipment, the malicious attacker can however acquire the sensitive information (passwords, business correspondence, etc.) readily to do real harm.

However, compared to the various works focusing on defenses against shoulder surfing, the works studying and modeling this threat are sparse and outdated. Most of these works focus on scenarios where the attacker peeks at the phone with his/her naked eye, performing experiments, or conducting surveys to research this threat. Eiband et al.~\cite{eiband2017understanding} conducted a user survey to investigate stories of shoulder surfing; Kwon et al.~\cite{kwon2013covert} designed a shoulder surfing approach called covert attentional shoulder surfing, and evaluated its success rates against PIN entry scenarios, presenting a powerful shoulder surfing threat model with the naked eye; and Schaub et al.~\cite{schaub2012password} evaluated shoulder surfing susceptibility, using Levenshtein distances and 7-point Likert scales to evaluate the accuracy and difficulties of shoulder surfing on different virtual keyboards. These works, however, lack quantitive modeling of this privacy threat, such as controlling the distances, illumination, angle, etc. to evaluate the vulnerability of screen privacy in different scenarios, and these works all focus on the unequipped attacker, so that their attack range is within 1 meter or even closer (where the attacker stands right behind or next to the victim), which is a barely practical scenario.

To deal with the threat model of tool-assisted shoulder surfing, Maggi et al. designed an automatic shoulder surfing threat model, observing the target smartphone with a digital camera~\cite{maggi2011poster}. However, the system contains only recognition algorithms, lacking SR processors to enhance the quality of the image, so that it can only function when the attacker is standing at close range. By tailoring and fusing the SR technology with smartphones possessing powerful lenses and processors, we propose a stronger shoulder surfing threat model in which attackers can deploy it on commercial smartphones while obtaining information for a longer range to reduce suspicion, say 1.8-6m away from the victim's screen with an observing view of 30$^\circ$ shown in Table~\ref{tbl:comparison}. Evaluations demonstrate its feasibility and privacy concerns in our daily life. We also performed thorough evaluations of our threat model, measuring its abilities with various environmental parameters. To the best of our knowledge, we are the first work to design and model the new form of shoulder surfing attack with the assistance of smartphones and multi-frame SR algorithms.
\begin{table}[t]
  \caption{A comparison of the state-of-the-arts on shoulder surfing.}
  \vspace{2mm}
  \begin{center}
  \begin{tabular}{ccccc}
  \hline
   Reference & Scenarios   & Metric & \begin{tabular}[c]{@{}l@{}}Quantitative \\ Model\end{tabular}& Distance$^{\mathrm{a}}$ \\
  \hline
  Eiband~\cite{eiband2017understanding}  & Naked eye & $\times$ & $\times$ & - \\
  Kwon~\cite{kwon2013covert}  & Naked eye & \cmark & $\times$ &1m \\
Schaub~\cite{schaub2012password} & Naked eye & \cmark & $\times$ & -$^{\mathrm{b}}$\\
  Maggi~\cite{maggi2011poster}  & Camera & \cmark  & $\times$ &-$^{\mathrm{b}}$ \\
  \hline
  \hline
  \textbf{\textsf{SRPeek}} & COTS phones & \cmark & \cmark  & 1.8 / 6m\\
  \hline
  \multicolumn{5}{l}{$^{\mathrm{a}}$The maximum distance to the victim's screen. $^{\mathrm{b}}$The attacker stands}\\

  \multicolumn{5}{l}{next to the victim without the maximum effective distance.}\\
  \end{tabular}
  \label{tbl:comparison}
  \end{center}
\end{table}

\subsection{Super Resolution}
Image Super Resolution is the process of reconstructing an image with a higher spatial resolution. Based on the structural patterns, self-similarity~\cite{suetake2008image}, or previous knowledge of the image genre, the single-image SR techniques take as input a single low-resolution image, rendering a sharp, high-resolution one by deducing missing information and reconstructing the missing pixels. Further, multi-image SR techniques work on a set of pictures on the same scene, such as multiple snapshots from a smartphone, successive images from a satellite, or adjacent frames on a video clip. These algorithms collect extra data from slight differences between these images, often exhibiting better performance than single-image SR algorithms.

Recent works on SR are mostly based on Convolutional Neural Networks(CNN)~\cite{dong2015image} and Generative Adversarial Networks(GAN)~\cite{ledig2017photo}, the former often gets closer to ground truth while the latter generates fewer artifacts and is more pleasing to the human eye. To resolve multi-image SR tasks, say video SR, some works~\cite{shi2016real,kappeler2016video} use 3-dimensional convolutions to utilize sequentiality and consistency between adjacent frames. Some works also modify the dataflow among the network layers to merge neighboring frames~\cite{huang2017video}, or recurrently process the frames under the guidance of the output of the previous frame~\cite{sajjadi2018frame}. For images without consistency or sequential information, like satellite images, most works choose hybrid methods, solving the multi-image SR problem with multiple single-image SR procedures. They either merge the results of single-image SR algorithms for efficiency~\cite{kawulok2019deep}, or build a multi-image network to create a comprehensive view based on single-image SR networks~\cite{8834937}. Due to the extreme blurriness of snapshots, these methods cannot deal with the new shoulder surfing threat model we proposed, which takes as input massive blurred snapshots without the consistency between frames. And they achieve limited performance, shown in Table~\ref{tbl:comparison}.

