\section{Related Work}
\label{sec-related-work}
\begin{table}[t]
  \caption{A comparison of the state-of-the-arts on shoulder surfing.}
  \vspace{2mm}
  \begin{center}
  \begin{tabular}{cccc}
  \hline
   Reference & Scenarios & Training & distance$^{\mathrm{a}}$ \\
  \hline
  Eiband et al.~\cite{eiband2017understanding}  & naked eye & trained & 1 meter \\
  Kwon et al.~\cite{kwon2013covert}  & naked eye & untrained  & Intimate$^{\mathrm{b}}$\\
Schaub et al.~\cite{schaub2012password} & naked eye & untrained & Intimate\\
  Maggi et al.~\cite{maggi2011poster}  & camera & untrained  & Intimate \\
  \hline
  \hline
  \textbf{\textsf{SRPeek}} & COTS phones & trained  & 1.8 / 6 meter\\
  \hline
  \multicolumn{4}{l}{$^{\mathrm{a}}$The maximum distance to the victim's screen.}\\

  \multicolumn{4}{l}{$^{\mathrm{b}}$The attacker stands next to or behind the victim.}\\
  \end{tabular}
  \label{tbl:comparison}
  \end{center}
\end{table}

\subsection{Shoulder Surfing}
With the arrival of the information era, privacy issues are becoming increasingly prominent. Smartphone screen privacy, the concern of our smartphones being observed by strangers in public areas, or shoulder surfing, have been studied heavily recently~\cite{eiband2017understanding,goucher2011look,kwon2013covert}. To mitigate this threat, some systems hide the information~\cite{brudy2014anyone} or warn the user~\cite{saad2018communicating} once sensing malicious passers-by; others modify the user interfaces, including creating honeypots (for passwords)~\cite{chakraborty2014tag}, confusing unauthorized parties~\cite{wiedenbeck2006design}, and making the interactions invisible~\cite{kumar2007reducing} or unreadable from a distance~\cite{Chun2019Keep}. Most of the works assume that the attacker is a casual passer-by, taking occasional peeks with the naked eye, as is the case most of the time~\cite{eiband2017understanding} ~\cite{wiedenbeck2006design}. Given the assisted equipment, the malicious attacker can however acquire the sensitive information (passwords, business correspondence, etc.) readily to do real harm. To deal with the threat model of tool-assisted shoulder surfing,%\cl{more citation and table 1 row-2},
% Schaub, et, al. observed the susceptibility of shoulder surfing from the naked eye\cite{schaub2012password} 
Maggi et al. designed an automatic shoulder surfing threat model, observing the target smartphone with a camera~\cite{maggi2011poster}. It can however only function when the attacker is standing at close range, which is a barely practical scenario. By tailoring and fusing the SR technology with smartphones with developed mobile camera module and processing ability, we propose a stronger shoulder surfing threat model in which attackers can deploy it on commercial smartphones while obtaining information for a longer range to reduce suspicion, say 1.8-6m away from the victim's screen with an observing view of 30$^\circ$ shown in Table~\ref{tbl:comparison}. Evaluations demonstrate its feasibility and privacy concerns in our daily life.

\subsection{Super Resolution}
Image SR is the process of reconstructing an image with a higher spatial resolution. Based on the structural pattern, self-similarity~\cite{suetake2008image}, or previous knowledge of the image genre, the single-image SR techniques take as input a single low-resolution image, rendering a sharp, high-resolution one by deducing missing information and reconstructing the missing pixels. Further, multi-image SR techniques work on a set of pictures on the same scene, such as multiple snapshots captured by a smartphone, successive images from a satellite, or adjacent frames on a video clip. These algorithms collect extra data from slight differences of these redundant images, often exhibiting better performance than single-image SR algorithms. And a variety of techniques have been proposed for the multi-image SR problem. And early works focus on the analysis of images in spatial or frequency domains. For example, the Shift-Add algorithm~\cite{farsiu2003robust} analyzes the spatial information by maximizing the pixel-wise possibility of high-resolution image from low-resolution images. Besides, approaches for the frequency domain rely on the Fourier transform of images. Assuming that the high-resolution scene is band-limited, these algorithms reconstruct high-frequency components from patterns of low-frequency components, removing bands of noise and blur effects simultaneously. Unfortunately, all SR techniques  however, are subject-specific and achieve the best performance only for some contents, including natural photos, scanned text, satellite imaging, biometrics~\cite{Youm2016Image}
%\cl{more citation and table 1 row-3}
. While \textsf{SRPeek} can recover multi-type characters with complex strokes, including Chinese, English and numbers.

Deep learning approaches can also be adopted for Super Resolution. For example, SRCNN~\cite{dong2015image} was designed using CNN by replacing pooling layers with upsampling layers, achieving notably better performance than previous approaches. It however only focuses on single image SR, as the neural network accepts a single image as input. Besides, Generative Adversarial Networks (GAN) were also used to generate photo-realistic images~\cite{ledig2017photo}, with less artifacts and more genuine-looking details perceptible to the human eye. To resolve multi-image SR tasks, say video SR, existing works~\cite{shi2016real, kappeler2016video} design a structure similar to SRCNN, and accept complete video clips as input. Then the sequentiality and consistency between adjacent frames can help recover the missing pixels of clips using convolution networks, Specifically,
% modifying the 2d convolution layers to 3d convolution~\cite{caballero2017real}, 
we can modify the dataflow to merge neighboring frames among the network layers~\cite{huang2017video}, or recurrently processing the frames under the guidance of the output of the previous frame~\cite{sajjadi2018frame}. For images without consistency or sequential information, like satellite images, most works choose hybrid methods to solve the multi-image SR problem with multiple single-image SR procedures. They either merge the results of single-image SR algorithms for efficiency~\cite{kawulok2019deep}, or build a multi-image network to create a comprehensive view based on single image SR networks~\cite{8834937}. Due of the extreme blurriness of snapshots, these methods can not deal with the new shoulder surfing threat model we proposed, which takes as input massive blurred snapshots without the consistency between frames. And they achieve limited performance, shown in Table~\ref{tbl:comparison}.
% These methods expect at least some degree of information to be extracted from each of the single frames with single image SR methods.