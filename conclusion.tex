
\section{Discussion}
\label{sec-discussion}
Although our shoulder surfing system proves to be highly efficient against unprotected screens, there are some simple methods to mitigate this unique threat while not cumbering the user.

\vspace{1mm}
\noindent
\textbf{Dynamic background.} Most multi-frame SR algorithms are designed based on the assumption that all input images are reflections of the same scene, and ours is not an exception. By deploying a dynamic background behind the characters, such as tiny dots and lines travelling slowly around the screen, we can construct a constantly changing scene that will confuse the multi-frame SR algorithms, and due to the blurriness of the images, these influencial elements cannot be easily removed. These dots do not need to be distinct or colored same as the the texts, as multi-frame SR algorithms function with tiny,pixel level differences between frames, making them especially sensitive to microscopic changes.

\textbf{Active scanning.} There are several works providing an active countermeasure against shoulder surfing threats with the naked eye. With front-facing cameras and face detection algorithms, the smartphone can constantly scan the surrounding passers-by and detect their gaze direction, and give a warning to the user when that gaze points at the screen. However, to the extent of our knowledge none of these works have included cameras into their detection scope, but we believe it's practical to implement such features.

\textbf{Adversarial machine learning methods.} In recent years we have discovered the weaknesses of neural networks and that inserting certain microscopic changes, undetectable to the human eye, to the pixels of an image will make it look different to a neural network. These methods fools the feature extraction phases of neural networks, so that SRPeek is also vulnerable to this attack. Theoretically, by exerting a pattern to the victim's screen, it can be captured by the attacker's camera and confuse its SR algorithms, but these remains to be implemented in our future work.

\section{Conclusion}
\label{sec-conclusion}
In this work we designed a holistic system, SRPeek, for shoulder surfing on smartphone, which serves as an up-to-date version of a threat model for shoulder surfing, and proved its efficiency. We proved that this threat towards screen privacy is imminent and can steal critical information, including personal texts or passwords, from long distances, thus escaping detection. It is our wish that this work can stur discussion in the field of screen privacy protection and proporgate defense mechanisms across critical mobile apps.

The core of SRPeek is a specially designed multi-frame SR network. With its innovative architecture this network outperforms other algorithms of the same field in our application. The design ideology enables this network to process higher levels of data integration ability while keeping a low calculation profile, and we believe the elements of this design can be used in other applications with large amounts of data, such as natural language processing or anomaly detection. Our model can also be used in OCR tasks when multiple images are avaliable, functioning as a preprocessor to improve the quality of the images and increase accuracy.