\section{Threat Model}
\label{sec-threat-model} 
Given the smartphone’s camera equipped with the SR network, the objective of the attacker is to acquire the multi-type of contents accurately on the smartphone screen of the victim from a long range, covering numbers, English and Chinese characters on screen. To reduce suspicion, the attacker will probably position himself at about 1.5 meters from the target screen, at an angle within 30 degrees. Since few people can be on guard of strangers at this distance even when viewing sensitive data or entering passwords. The attacker may raise his phone (one of the latest models with powerful lenses and computational abilities), pretending to be interacting with it, while pointing his camera at the victim. He will extend his zooming ability to the maximum, try to get a focus on the screen (which is difficult to achieve), and take 20 to 30 images in burst mode, with about 0.05 ~ 0.1 seconds of interval between frames. This process will take about 2 seconds, and we assume the information on the screen will not change in this period, nor the position of the screen. These images will be fed to a multi-frame SR network and the result, one single image with higher resolution, will be displayed on the screen soon afterwards. The characters in the image will be reconstructed to the best of the network’s ability, and if successful, the attacker will be able to decipher the information. The above procedure can be automated by an app and repeated every few seconds, giving the attacker a continuous surveillance over the victim’s screen.
