\section{Limitations}
\subsection{Image Capturing Ability}
Latest models of smartphones can capture images at 10 frames per second in burst mode easily, however, this ability is not common in phones that are 3 years old. Heavily used phones also performs less than ideal when capturing images in burst mode. Also, when capturing images the user need to hold their phone as still as possible to avoid motion blur and assist image alignment. The user also has to keep a sharp focus on the target phone.
\subsection{Processing Ability}
To achieve best performance the user needs a phone with strong processing capabilities to run the neural network at real time. As neural networks have been common place in numerous modern APPs, most phones of the latest generation have upgraded their processing ability to run neural networks, but older versions might not possess such processing powers and cannot process images at real-time.
\subsection{Motion and Tremors}
In our experiment we assume the observed user will hold still his/hers phone, and not making interactions too often. However some users may tilt their phones during interactions—especially when typing with one hand. Too much tremor will cause motion blur in captured images and misalignment between frames, thus degrading the result.